%%-----------------------------------------------------------------------
%% Make your own quadrille, graph, hex, etc paper!
%% Uses the pgf/TikZ package for LaTeX, which should be part of
%% any modern TeX installation.
%% Email: mcnees@gmail.com
%% Twitter: @mcnees
%%-----------------------------------------------------------------------

\documentclass[11pt]{article}

%%-----------------------------------------------------------------------
%% This package gives small margins so you use less
%% paper.
%%-----------------------------------------------------------------------
\usepackage[cm]{fullpage}

%%-----------------------------------------------------------------------
%% Packages needed for the drawing routines and for mixing custom
%% colors.
%%-----------------------------------------------------------------------
\usepackage{tikz}
\usetikzlibrary{patterns}
\usepackage{xcolor}


%%-----------------------------------------------------------------------
%% Some nice colors.
%%-----------------------------------------------------------------------
\definecolor{plum}{rgb}{0.36078, 0.20784, 0.4}
\definecolor{chameleon}{rgb}{0.30588, 0.60392, 0.023529}
\definecolor{cornflower}{rgb}{0.12549, 0.29020, 0.52941}
\definecolor{scarlet}{rgb}{0.8, 0, 0}
\definecolor{brick}{rgb}{0.64314, 0, 0}
\definecolor{sunrise}{rgb}{0.80784, 0.36078, 0}


%%-----------------------------------------------------------------------
%% Colors used for the minor and major grids.
%%-----------------------------------------------------------------------
%% A light blue color for the minor grid.
\colorlet{minorcolor}{cornflower!35}
%% A darker blue color for the major grid and/or frame.
\colorlet{majorcolor}{cornflower!60}
%% You can make either one darker by changing the number after the "!".
%% For instance, cornflower!75 would be much darker.


%%-----------------------------------------------------------------------
%% Comment out the previous lines and uncomment the following lines for
%% graph paper with a light green color, like on an engineering pad.
%%-----------------------------------------------------------------------
% \colorlet{minorcolor}{chameleon!30}
% \colorlet{majorcolor}{chameleon!50}

%%-----------------------------------------------------------------------
%% This section sets up a routine for filling a shape with
%% hexagons. Uses code from:
%% http://tex.stackexchange.com/questions/6019/drawing-hexagons/6128#6128
%%-----------------------------------------------------------------------

  % Big hexagons
  %\def\hexagonsize{0.1666in}
  % Small hexagons
  \def\hexagonsize{0.0833in}

  \pgfdeclarepatternformonly
    {hexagons}% name
    {\pgfpointorigin}% lower left
    {\pgfpoint{3*\hexagonsize}{0.866025*2*\hexagonsize}}
    {\pgfpoint{3*\hexagonsize}{0.866025*2*\hexagonsize}}
    {
     \pgfsetlinewidth{0.6pt}
     \pgftransformshift{\pgfpoint{0mm}{0.866025*\hexagonsize}}
     \pgfpathmoveto{\pgfpoint{0mm}{0mm}}
     \pgfpathlineto{\pgfpoint{0.5*\hexagonsize}{0mm}}
     \pgfpathlineto{\pgfpoint{\hexagonsize}{-0.866025*\hexagonsize}}
     \pgfpathlineto{\pgfpoint{2*\hexagonsize}{-0.866025*\hexagonsize}}
     \pgfpathlineto{\pgfpoint{2.5*\hexagonsize}{0mm}}
     \pgfpathlineto{\pgfpoint{3*\hexagonsize+0.2mm}{0mm}}
     \pgfpathmoveto{\pgfpoint{0.5*\hexagonsize}{0mm}}
     \pgfpathlineto{\pgfpoint{\hexagonsize}{0.866025*\hexagonsize}}
     \pgfpathlineto{\pgfpoint{2*\hexagonsize}{0.866025*\hexagonsize}}
     \pgfpathlineto{\pgfpoint{2.5*\hexagonsize}{0mm}}
     \pgfusepath{stroke}
    }

%%-----------------------------------------------------------------------
%% This section sets up a routine for filling the squares in a
%% grid with null lines.
%%-----------------------------------------------------------------------
  \def\squaresize{0.25in}
  \pgfdeclarepatternformonly
    {lightcones}% name
    {\pgfpointorigin}% lower left
    {\pgfpoint{\squaresize}{\squaresize}}%  upper right
    {\pgfpoint{\squaresize}{\squaresize}}%  tile size
    {% shape description
     \pgfsetlinewidth{0.4pt}
     \pgfpathmoveto{\pgfpoint{0in}{0in}}
     \pgfpathlineto{\pgfpoint{\squaresize}{\squaresize}}
     \pgfpathmoveto{\pgfpoint{0in}{\squaresize}}
     \pgfpathlineto{\pgfpoint{\squaresize}{0in}}
     \pgfusepath{stroke}
    }

%%-----------------------------------------------------------------------
%% This section sets up a routine for filling a region with dots
%%-----------------------------------------------------------------------
  % Re-use the quantity \squaresize defined above
  \def\dotsize{.7pt}
  \pgfdeclarepatternformonly
    {dotgrid}% name
    {\pgfpoint{-0.5*\squaresize}{-0.5*\squaresize}}% lower left
    {\pgfpoint{0.5*\squaresize}{0.5*\squaresize}}%  upper right
    {\pgfpoint{\squaresize}{\squaresize}}%  tile size
    {% shape description
     \pgfpathcircle{\pgfqpoint{0pt}{0pt}}{\dotsize}
     \pgfusepath{fill}
    }

\begin{document}

% No page numbers!
\thispagestyle{empty}

%%-(Section)-------------------------------------------------------------

%%-----------------------------------------------------------------------
%% Use this section for quadrille, ten squares per inch.
%%-----------------------------------------------------------------------
  % \begin{tikzpicture}[remember picture, overlay]
  %
  %   % Change "very thin" to "thin" if the lines are too thin.
  %   \tikzset{
  %     minorgrid/.style={minorcolor, thin},
  %     majorgrid/.style={majorcolor, thin},
  %   }
  %
  %   % Draw a grid with 10 squares per inch.
  %   \draw[style=minorgrid, shift={(current page.south west)},shift={(0.2in,0.2in)}] (0,0) coordinate (a) grid
  %   [step=0.1in] (8.1in,10.6in) coordinate (b);
  %
  %   % Draw a frame around the grid.
  %   \draw[style=majorgrid] (a) rectangle (b);
  %
  %
  % \end{tikzpicture}

%%-(Section)-------------------------------------------------------------

%%-----------------------------------------------------------------------
%% Use this section for quadrille, eight squares per inch.
%%-----------------------------------------------------------------------
  %   \begin{tikzpicture}[remember picture, overlay]
  %
  %     % Change "very thin" to "thin" if the lines are too thin.
  %     \tikzset{
  %       minorgrid/.style={minorcolor, thin},
  %       majorgrid/.style={majorcolor, thin},
  %     }
  %
  %     % Draw a grid with 10 squares per inch.
  %     \draw[style=minorgrid, shift={(current page.south west)},shift={(0.1875in,0.1875in)}] (0,0) coordinate (a)
  % grid [step=0.125in] (8.125in,10.625in) coordinate (b);
  %
  %     % Draw a frame around the grid.
  %     \draw[style=majorgrid] (a) rectangle (b);
  %
  %   \end{tikzpicture}

%%-(Section)-------------------------------------------------------------

%%-----------------------------------------------------------------------
%% Use this part for graph, eight squares per inch with a major grid
%% every half-inch.
%%-----------------------------------------------------------------------
  % \begin{tikzpicture}[remember picture, overlay]
  %
  %   % Change "very thin" to "thin" if the lines are too thin.
  %   \tikzset{
  %     minorgrid/.style={minorcolor, thin},
  %     majorgrid/.style={majorcolor, thin},
  %   }
  %
  %   % Draw a grid with 10 squares per inch.
  %   \draw[style=minorgrid, shift={(current page.south west)},shift={(0.225in,0.25in)}] (0,0) coordinate (a) grid [step=0.125in] (8in,10.5in) coordinate (b);
  %
  %   \draw[style=majorgrid, shift={(current page.south west)},shift={(0.225in,0.25in)}] (0,0) coordinate (a) grid [step=0.5in] (8in,10.5in) coordinate (b);
  %
  %   % Draw a frame around the grid.
  %   \draw[style=majorgrid] (a) rectangle (b);
  %
  % \end{tikzpicture}

%%-(Section)-------------------------------------------------------------

%%-----------------------------------------------------------------------
%% Use this section to draw a hex grid on the page.
%%-----------------------------------------------------------------------
  % \begin{tikzpicture}[remember picture,overlay]%
  %   \node at (current page.north west) [anchor=north west, inner sep=0pt, outer sep=0.0in]{
  %     \tikz{
  %     \fill [pattern=hexagons,pattern color=minorcolor] (0,0) rectangle (8.5in,11in);
  %       }
  %   };
  % \end{tikzpicture}

%%-(Section)-------------------------------------------------------------

%%-----------------------------------------------------------------------
%% Use this section to draw a dot grid on the page.
%%-----------------------------------------------------------------------
  \begin{tikzpicture}[remember picture,overlay]%
    \node at (current page.north west) [anchor=north west, inner sep=0pt, outer sep=0.0in]{
      \tikz{
      \fill [pattern=dotgrid,pattern color=minorcolor] (0,0) rectangle (8.5in,11in);
        }
    };
  \end{tikzpicture}

%%-(Section)-------------------------------------------------------------

% %%-----------------------------------------------------------------------
% %% Use this section to draw a grid with light cones.
% %%-----------------------------------------------------------------------

  % % The color to use for the null directions.
  % \colorlet{lightlines}{scarlet!30}

  % % Colors and line style for the squares and the border.
  % \tikzset{
  %   minorgrid/.style={cornflower!30, thin},
  %   majorgrid/.style={cornflower!50, thick},
  % }

  % \begin{tikzpicture}[remember picture, overlay]
  %   % Draw a grid with 4 squares per inch.
  %   \draw[style=minorgrid, shift={(current page.south west)}] (0.25in,0.25in) coordinate (a) grid
  % [step=0.25in] (8.25in,10.75in) coordinate (b);

  %   % Draw a border around the grid.
  %   \draw[style=majorgrid] (a) rectangle (b);

  %   % Now fill the grid with 45 degree lines in the color defined for null directions.
  %   \node at (current page.south west) [anchor=south west, inner sep=0, style=majorgrid, shift={(a)}]{
  %   \tikz{
  %       \fill [pattern=lightcones,pattern color=lightlines] (0.25in,0.25in) rectangle (8.25in,10.75in);
  %     }
  %   };

  % \end{tikzpicture}

%%-(End)-------------------------------------------------------------


\end{document}